\section[\hspace{-0.42cm}Abstracts
 \dotfill{}]{}
\vspace{-.56cm}
\noindent {{\rm  {\bf ABSTRACTS}}}

% {\it \Large \noindent Compressive Sensing}

\section[\hspace{-0.42cm}Prodanov~T.~P.~(SPb) Adaptive Randomized Algorithms for Community Detection in Graphs\dotfill{}]{}
\vspace{-.56cm}
\noindent {\Large{\rm  {\bf Adaptive Randomized Algorithms\\ for Community Detection in Graphs}}}%\footnote{%\copyright T.~P.~Prodanov, 2015}
\\   {\it T.~P.~Prodanov}
\\  {\it St. Petersburg State University}
\\ timofey.prodanov\symbol{64}gmail.com

%\vspace{-1.5cm}

\noindent \hrulefill{}

\vspace{0.2cm}

\noindent {\it Key words\/}: community detection, complex networks, stochastic approxi\-mation.

\vspace{0.1cm}

Last seventeen years the study of complex networks grows rapidly and search of tighly connected groups of nodes, or community detection, proved to be powerful tool for analyzing real systems.

Randomized algorithms for community detection are effective but there is no parameters that make these algorithms create good partitions for every input complex network. In article to solve this problem I apply stochastic approximation to two randomized algorithms and propose two new adaptive modifications that adjust parameters to input data and create good partitions for wider range of input networks.

Bibliogr.: 26 refs.
